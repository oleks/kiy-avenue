\section{Documentation Generators}

\subsection{Markup}

Most documentation generators seem to employ some sort of a markup language for
annotation. We distinguish between the following markup language types:

\begin{description}[\setleftmargin{90pt}\setlabelstyle{\bf}]

\item [Tag] Tag-based markup languages use (named or unnamed) begin- and
end-tags to annotate comment strings; for instance, XML-based markup, or \TeX{}
groups.

\item [Keyword] Keyword-based markup languages use keywords to annotate comment
strings. A keyword initialises an annotation, while a special charater, e.g.
line break, ends the said annotation.

A keyword may also initialise a sequence of annotations (including the empty
sequence). For instance, the \texttt{@param} annotation in javadoc, consumes
the subsequent non-broken string as a parameter name, before consuming the
subsequent non-line-broken string as a comment for that parameter.

\item [Mixed] An instructive example of a mixed markup language is \TeX{}.

\end{description}

Of course, this categorisation is rather superfluous, as keyword-based markup
is just a special case of tag-based markup. Keywords that initialise a sequence
of annotations can be seen as equivalent to tags with particular mandatory
children tags.

\subsection{Generators}

\subsubsection{javadoc}

javadoc is a documentation generator for the
Java\textsuperscript{\texttrademark} programming language. It employs a
keyword-based mark-up language for the documentation of the API exposed by Java
source code. The output of the generator is a set of hyperlinked static HTML
pages.

The documentation hierarchy is \begin{inparaenum}[(1)] \item fields,
constructors and methods, \item classes, interfaces and enums, and \item
packages\end{inparaenum}. Most mark-up constructs concern themselves with the
first element of the hierarchy, describing the uses of the fields; the
parameters, assumptions, guarantees and failure scenarios of methods (where
constructors are but fancy methods).

javadoc makes no assumptions about the source code domain, beyond the
underlying programming language being Java. The documentation generator is not
easily extensible with new mark-up constructs. Although it is possible, it
requires programming against the javadoc API --- a tedious task for the common
administrator or project leader. Nor does the documentation generator allow to
enforce documentation standards, such as mandatory mark-up of the methods' time
and space complexities.

\subsubsection{Sphinx}

\subsubsection{Doxygen}

\subsubsection{Sandcastle}

\subsubsection{Red Gate SQL Doc}

Red Gate SQL Doc is a commercial documentation generator for SQL-based databases. 

\subsubsection{Haddock}

\subsubsection{man}

man is a manual paging system for *nix systems.

\subsubsection{Pandoc}
