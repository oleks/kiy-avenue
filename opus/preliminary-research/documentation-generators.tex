\section{Documentation Generators}

\subsection{Markup}

Most documentation generators seem to employ some sort of a markup language for
annotation. We distinguish between the following markup language types:

\begin{description}[\setleftmargin{90pt}\setlabelstyle{\bf}]

\item [Tag] Tag-based markup languages use (named or unnamed) begin- and
end-tags to annotate comment strings; for instance, XML-based markup, or \TeX{}
groups.

\item [Keyword] Keyword-based markup languages use keywords to annotate comment
strings. A keyword initialises an annotation, while a special charater, e.g.
line break, ends the said annotation.

A keyword may also initialise a sequence of annotations (including the empty
sequence). For instance, the \texttt{@param} annotation in javadoc, consumes
the subsequent non-broken string as a parameter name, before consuming the
subsequent non-line-broken string as a comment for that parameter.

\item [Mixed] An instructive example of a mixed markup language is \TeX{}.

\end{description}

Of course, this categorisation is rather superfluous, as keyword-based markup
is just a special case of tag-based markup. Keywords that initialise a sequence
of annotations can be seen as equivalent to tags with particular mandatory
children tags.

\subsection{Generators}

\subsubsection{javadoc}

\subsubsection{Sphinx}

\subsubsection{Doxygen}

\subsubsection{Sandcastle}

\subsubsection{Red Gate SQL Doc}

\subsubsection{Haddock}

\subsubsection{man}
